\documentclass{beamer}

\input{../tools/beamerpreamble.tex}

\title[GB21802]{GB21802 - プログラミングチャレンジ}
\subtitle[]{Week 0 - イントロダクション}
\author[Claus Aranha]{Claus Aranha\\{\footnotesize caranha\@@cs.tsukuba.ac.jp}}
\institute{Department of Computer Science}
\date{2017/4/14\\{\smaller(last updated: \today)}}

\begin{document}

\section{イントロダクション}
\subsection{授業について}

\begin{frame}
\maketitle
\end{frame}


\begin{frame}
  \frametitle{まず初めに: 大切なお知らせ}

  \begin{block}{Manaba ページ}
    この授業のすべての講義ノートとお知らせは MANABA で行われます。
    下の URL からアクセスしてください。
    
    \medskip
    
    \url{https://manaba.tsukuba.ac.jp/ct/course_781339}\\
    登録コード: 8467527
  \end{block}
  \begin{exampleblock}{言語}
    \begin{itemize}
      \item 講義: 日本語
      \item スライドと資料: 英語
      \item 練習問題: 英語
      \item 質問、メール、宿題: 好きな言語
    \end{itemize}
  \end{exampleblock}
\end{frame}

\begin{frame}
  \frametitle{講義担当について}
  \begin{columns}
    \column{0.4\textwidth}
    \includegraphics[width=1\textwidth]{../img/pinhole}
    \column{0.6\textwidth}
    {\small
    \begin{itemize}
      \item \structure{名前:} Claus Aranha (アランニャ・クラウス);
      \item \structure{出身国:} ブラジル;
        
        \medskip

      \item \structure{研究領域:} 人工知能、遺伝的アルゴリズム、ディープラーニング
      \item \structure{プログラミング言語:} Python、R
      \item \structure{趣味:} ゲームプログラミング、ジオキャッシング、Twitter
        Bots
        
        \medskip

      \item \structure{twitter:} \href{http://www.twitter.com/caranha}{@caranha}
      \item \structure{ウェブページ:}\\ {\smaller \url{http://conclave.cs.tsukuba.ac.jp}}
    \end{itemize}
    }
  \end{columns}
\end{frame}

\begin{frame}
  \frametitle{この講義で扱う対象は?}
  
  これまでに皆さんは、たくさんのプログラミングテクニックを学習してきたと思います...\\\hfill ...でも、ちゃんと使いこなせていますか?

  \begin{block}{授業の基本指針: Learning by Practice}
    \begin{itemize}
      {\small
    \item 毎週、いくつかの問題を解くことを求められます。
    \item 問題を解くためには、最適な\structure{データ構造}と
      \structure{アルゴリズム}を選択しなければなりません。
    \item また、各問題には、 \alert{制限時間}と\alert{メモリ制限}が存在します。
    \item アルゴリズムやテクニック、トリックについて講義します。
      }
    \end{itemize}
  \end{block}

  \begin{exampleblock}{授業の目的:}
    プログラミングの能力やテクニックを磨き、
    プログラミングに慣れ親しむこと。
  \end{exampleblock}
\end{frame}

\begin{frame}
  \frametitle{このコースを受けるのにふさわしい人}
  \begin{itemize}
  \item \structure{プログラムを作るのが好きな人、プログラミングが楽しいと思っている人}

    \smallskip
    
  \item プログラミングの理論はたくさん勉強したけれど、練習不足を感じている人。
    
    \smallskip

  \item まだプログラムをたくさん書いた経験がない人

    \smallskip

  \item プログラムの効率について考えられるようになりたい人。

    \smallskip

  \item 記憶力よりスキルが重視される授業を受けたい人。

    \smallskip

  \item テクニカルな英語の練習がしたい人。

    \smallskip

  \item プログラミングコンテストに出場したい人。
  \end{itemize}
\end{frame}

\begin{frame}
  \frametitle{この授業についての警告}
  \begin{alertblock}{1- Heavy Workload}
    \begin{itemize}
    \item Challenges start easy, but end very hard;
    \item Expect to use a few hours per week on homework;
    \item Lots of debugging;

      \bigskip

    \item Hint: Do your homework early!
    \end{itemize}
  \end{alertblock}

  \begin{alertblock}{2- 授業で使う言語}
    \begin{itemize}
    \item コースの資料はすべて英語で書かれています。
    \item 特に重要なポイント: 宿題もすべて英語です。
    \item ただし、提出するプログラムや質問は日本語でも大丈夫です。

      \bigskip

    \item この授業では、英語の練習もしてもらえたら嬉しいです! :-)
    \end{itemize}
  \end{alertblock}

\end{frame}

\section{プログラミングチャレンジ}
\subsection{例}

\begin{frame}
  \frametitle{``プログラミングチャレンジ''ってなに?}

  プログラミングチャレンジとは、コンピュータプログラムを作ることで解くことができるパズルです。

  \bigskip

  チャレンジには、\structure{入力}と問題の  \structure{ルール}が書かれていて、
  挑戦者は、\structure{正しい出力} を見つけるプログラムを書く必要があります。

  \bigskip

  例を見てみましょう。
  
\end{frame}

\begin{frame}
  \frametitle{チャレンジの一例: ``Relational Operator (関係演算子)'' (1)}

  {\small
  The challenges for this course are listed at the page:\\
  {\smaller \url{http://conclave.cs.tsukuba.ac.jp/lecture/monitor.html}}

  \begin{center}
    \includegraphics[width=.7\textwidth]{../img/monitorpage}
  \end{center}

  Click on the title to go to the problem page.}
\end{frame}

\begin{frame}
  \frametitle{Example Challenge: ``Relational Operator'' (2)}

  Clicking on the title will take you to the problem page.
  
  \begin{center}
    \includegraphics[width=.9\textwidth]{../img/relationaloperator}
  \end{center}

  Here you can read the problem and submit a solution.\\
  (You will need an UVA account!)

\end{frame}

\begin{frame}
  \frametitle{Example Challenge: ``Relational Operator'' (3)}
  
  \begin{block}{Problem Description}
    {\small 
      Some operator checks about the relationship between two values,
      these operators are called relational operators. Given two
      numerical values, your job is just to find out the relationship
      between them. That is (i) First one is greater than the second,
      (ii) First one is less than the second or (iii) First and second
      one is equal.
    }
  \end{block}
  \begin{block}{Input}
    {\small
      First line is the number $t$ of tests ($t < 15$). Following t lines 
      are two integers $a$ and $b$.
    }
  \end{block}
  \begin{block}{Output}
    {\small
      For each line of input, print one line of output with '>','<' or '=', 
      according to the relationship of $a$ and $b$.
    }
  \end{block}
\end{frame}



\begin{frame}[fragile]
  \frametitle{Solving ``Relational Operator''}
    
\begin{block}{}
{\smaller
\begin{verbatim}
// UVA 11172 - Relational Operator
// Test if a is bigger, smaller or equal to b

#include <iostream>
using namespace std;

int main()
{
    int n; long a, b;

    cin >> n;    
    for (; n > 0; n--)
    {
        cin >> a >> b;
        if (a > b) cout << ">\n";
        if (a < b) cout << "<\n";
        if (a == b) cout << "=\n";
    }
}
\end{verbatim}}
\end{block}
\end{frame}

\subsection{Submission}

\begin{frame}
  \frametitle{プログラムの提出の仕方} 
  
  After you finish your programs, \alert{and make sure they are
    correct}, you can submit it.

  \bigskip

  Your weekly routine should have four steps:
  \begin{enumerate}
    \item Think about how to solve each problem;
    \item Submit each problem to the UVA website, and check it is correct;
    \item Prepare your MANABA package (code + comment file);
    \item Submit your MANABA package to MANABA;
  \end{enumerate}
\end{frame}

\begin{frame}
  \frametitle{Submitting the problem to UVA (1)}

  {\small
  UVA is an \structure{Automated Robotic Judge}. It will test your
  program on a set of inputs, and check if the outputs are correct.

  From the problem page, click on the \structure{submit} button. 
  
  \begin{center}
    \includegraphics[width=.8\textwidth]{../img/submitpage}
  \end{center}
  
  Select your language, choose the file, and press submit.\\
  (You can use C, C++, Java, Python {\tiny and Pascal})}
\end{frame}

\begin{frame}
  \frametitle{Submitting the problem to UVA (2)}

  {\small
    After you submit the program, the judge will output one of the
    following results: Accepted, Wrong Answer, Time Limit Exceeded,
    Memory Limit Exceeded, Runtime Error, etc.
    
    \begin{center}
      \includegraphics[width=.8\textwidth]{../img/submissionpage}
    \end{center}

    You can see this information on the ``my submissions'' page.
  }
\end{frame}

\begin{frame}
  \frametitle{Submission Statues:}
  \begin{itemize}
  \item \structure{Accepted}: Your program is correct!
    Congratulations!
  \item \structure{Wrong Answer}: Your program is incorrect. Debugging
    time.
  \item \structure{Time/Memory limit exceeded}: Your program is
    inefficient. Think more.
  \item \structure{Runtime Error}: Your program is crashing. To the
    debugger!
  \end{itemize}

  \bigskip

  We will see how to deal with some of these problems in the next class.
\end{frame}

\begin{frame}
  \frametitle{Back to the problem Monitor}

  In the problem monitor page, you can check how many people solved
  each problem, which problems you still have to solve, and the
  deadlines.

  \begin{center}
    \includegraphics[width=.7\textwidth]{../img/monitorpage2}
  \end{center}
\end{frame}

\subsection{Manaba Submission}
\begin{frame}
  \frametitle{Submitting the problem to MANABA}

  {\small
  After you finish the problems listed in the monitor, you need to
  submit your source code and a comment file as a zip package to MANABA.}

  \medskip

  {\small
  \begin{columns}
    \column{0.3\textwidth}
    \column{0.4\textwidth}
    \begin{block}{s2015XXXXXX-weekYY.zip}
      \begin{itemize}
      \item problem1.cpp
      \item problem2.cpp
      \item problem5.cpp
      \item kaisetsu.txt
      \end{itemize}
    \end{block}
    \column{0.3\textwidth}
  \end{columns}
  }

  \medskip
  
  \begin{alertblock}{Attention}
  Submission to the UVA judge without a submission to MANABA will not
  be accepted!
  \end{alertblock}
\end{frame}

\begin{frame}
  \frametitle{Some warnings about Java:}

  \begin{itemize}
  \item All code must be in the same source file (can define many
    classes in this file)

    \medskip

  \item All programs must begin in a static main method in a
    \structure{Main} class.

    \medskip

  \item Do not use public classes. Even Main must be non public.

    \medskip

  \item Use Buffered I/O to avoid time limit exceeded.
  \end{itemize}
\end{frame}

\section{Course Rules}
\subsection{Course Structure}

\begin{frame}
    \frametitle{Outline}
    
    \begin{block}{Two classes per week}
        \begin{itemize}   
        \item Each week has a theme
        \item Friday Class: Introduction
        \item Monday Class: Problem Solving and Q\&A
        \end{itemize}
    \end{block}
    
    \begin{block}{Solving Problems}
        \begin{itemize}
        \item Every week there are 6-10 programming assignments;
        \item Assignments follow the weekly theme;
        \item Automatic Submission and Evaluation System;
        \item Program Deadline is Thursday 23:59
        \end{itemize}
    \end{block}
\end{frame}
    
\begin{frame}
  \frametitle{Outline}
  \begin{center}
    \includegraphics[width=1\textwidth]{../img/classoutline}
  \end{center}
\end{frame}

\subsection{Grading}

\begin{frame}
  \frametitle{Evaluation and Grading (1)}

  Evaluation Criteria: \structure{Problems solved}, \structure{Code}
  and \structure{Participation}
  
  \bigskip

  Evaluation Process: Base Grade \structure{+Bonus} \alert{-Penalty}
\end{frame}

\begin{frame}
  \frametitle{Evaluation and Grading (2) -- Base Grade}

  The \structure{Base Grade} is based on homework submissions to UVA.

  \bigskip
  
  \begin{itemize}
  \item \structure{C}: One problem per lecture, or $X_c$ problems total

    \medskip

  \item \structure{B}: Two problems per lecture, or $X_b$ problems total

    \medskip

  \item \structure{A}: Three problems per lecture, or $X_a$ problems total
  \end{itemize}

  \vfill

  {\small
  Parameters $X_{a,b,c}$ will be decided at a later date. \alert{Do not rely on this}}

\end{frame}

\begin{frame}
  \frametitle{Evaluation and Grading (3) -- Bonus and Penalty}
  
  {\small
  A \structure{Bonus} or \structure{Penalty} will be added to the base grade.
  \begin{itemize}
    \item Bonus: grade one step up (C->B, B->A, A->A+)
    \item Penalty: grade one step down (A+->A, B->C, \alert{C->C})
  \end{itemize}
  }

  \medskip
  \begin{exampleblock}{Bonus: Grade Up}
    \begin{itemize}
    \item Participation in class and MANABA
    \item Submit corrections/suggestions to lecture notes
    \item Consistently good Comment/Kaisetsu file
    \item Best $N$ students in number of submissions
    \end{itemize}
  \end{exampleblock}
  \begin{alertblock}{Penalty: Grade Down}
    \begin{itemize}
    \item More than 25\% problems submitted after the deadline
    \end{itemize}
  \end{alertblock}

  \tiny{Parameter $N$ will be decided at a later date.}
\end{frame}

\begin{frame}[fragile]
  \frametitle{Evaluation and Grading (4) -- comment/kaisetsu file}

  When you submit your package every week, include a text file (no
  Word!) with comments on each problem you tried to solve.
    
  \begin{exampleblock}{Example}
    {\smaller
\begin{verbatim}
Name: Claus, ID: 98884735
# Problem 1:
To solve this problem, I sorted the input data, and 
printed the input with the highest number of repeated 
letters.

# Problem 2: 
I tried to solve this problem with brute force, but 
the time limit was exceeded. I had to use DP on the 
number of people instead.
\end{verbatim}}
  \end{exampleblock}

  \bigskip
  
  Comments may be in Japanese. {\small (FILENAMES must be in romaji)}
\end{frame}

\begin{frame}
  \frametitle{Evaluation and Grading (5) -- about plagiarism}
  
  The assignments are \alert{individual}. Use your \structure{own
    strength} to solve the programs.

  \begin{exampleblock}{GOOD}
    \begin{itemize}
    \item Ask for ideas to your friends;
    \item Ask for ideas in the MANABA forum;
    \item Ask for help with a bug;
    \end{itemize}
  \end{exampleblock}

  \begin{alertblock}{BAD}
    \begin{itemize}
    \item Copy a solution from the internet;
    \item Copy a solution from your friends;
    \item Give your code to a friend;
    \end{itemize}
  \end{alertblock}

  Plagiarism will result in course failure, and possibly worse.
\end{frame}

\section{Resources}
\subsection{Resources}

% Class Links
\begin{frame}
  \frametitle{Useful Links}
  \begin{itemize} 
  \item
    \href{https://manaba.tsukuba.ac.jp/ct/course_781339}
         {\structure{\underline{Manaba Page}}}: All the class material
         will be here. Access Code is: 8467527

    \medskip

  \item \href{https://uva.onlinejudge.org/}{\structure{\underline{UVA Online Judge}}}:
    Use this page to submit your problems. \alert{Make an account and list the username on MANABA}

    \medskip

  \item \href{https://conclave.cs.tsukuba.ac.jp/lecture/monitor.html}{\structure{\underline{Problem Monitor}}}:
    Use this page to check deadlines and weekly problems.

    \medskip

  \item
    \href{https://www.github.com/caranha/ProgrammingChallengesLectureNotes}{\structure{\underline{Github
          Repository}}}:
    Working directory for lecture notes. Send me PR, issues!

    \medskip
    
  \item
    \href{https://www.udebug.com/}{\structure{\underline{uDebug}}}:
    Web service that generates test inputs and test outputs for UVA
    problems. Useful tool for this course.
  \end{itemize}
\end{frame}


% Books
% TODO: Add images for the books (week 0A)
% TODO: Add japanese books
\begin{frame}
  \frametitle{Books}

  \begin{itemize}
  \item \structure{Main Book:} Competitive Programming, 3rd Edition
    \href{http://cpbook.net/}{Link}
    
    \bigskip

  \item \structure{Old Course Book:} Programming Challenges
    \href{https://books.google.co.jp/books/about/Programming_Challenges.html?id=dNoLBwAAQBAJ&source=kp_cover&redir_esc=y}{Link}

    \bigskip
    
  \item For suggestions of books in Japanese, please check the Manaba materials!
  \end{itemize}
\end{frame}

% udebug
\begin{frame}
  \frametitle{uDebug Tool}

  If you are having problems, the uDebug site offers, for many
  problems in UVA, the correct set of outputs for any input you give.

  \bigskip

  \url{https://www.udebug.com/}

  \bigskip

  \begin{center}
    \includegraphics[width=0.7\textwidth]{../img/udebug}
  \end{center}
\end{frame}

% Cat stream
\begin{frame}
  \frametitle{If you are still having problems...}

  Watch a cat stream to relax!

  \bigskip

  \begin{center}
    \includegraphics[width=.6\textwidth]{../img/catstream}
  \end{center}

  \bigskip

  \url{http://livestream.com/FosterKittenCam/}
\end{frame}


% Contact the professor (e-mail, twitter, webpage, room)
\begin{frame}
  \frametitle{Contact the professor}
  \begin{itemize}
  \item \structure{e-mail}: caranha@cs.tsukuba.ac.jp
  \item \structure{website}: \url{http://conclave.cs.tsukuba.ac.jp}
  \item \structure{twitter}: @caranha

    \bigskip

  \item \structure{Room}: SB1012\\
    Best times to find me:
    \begin{itemize}
    \item Morning (9:00 -- 11:00): Monday, Wednesday, Friday
    \item Evening (17:00 -- 19:00): Monday, Tuesday, Wednesday
    \end{itemize}
  \end{itemize}

  \bigskip

  Both English and Japanese are okay!
\end{frame}


% Do we still have time? Fill out the forms!
% Do we still have time? Ask me questions!

\begin{frame}
  \frametitle{Do we still have some time?}

  \begin{itemize}
  \item Create an account on UVA (if you already have an account, you can use that)

    \bigskip

  \item Submit your account name to the MANABA

    \bigskip

  \item Ask any other questions you want to know!
  \end{itemize}

  \bigskip

  \begin{center}
    Thank you for today!
  \end{center}
\end{frame}

\end{document}
